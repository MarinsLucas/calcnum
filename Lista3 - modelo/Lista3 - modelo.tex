\documentclass{article}
\usepackage[utf8]{inputenc}
\usepackage[brazil]{babel}
\usepackage{epsfig}
\usepackage{fancyhdr}
\usepackage{indentfirst} 
\usepackage{titlesec}
\usepackage{amsmath}
\usepackage{amsthm}
\usepackage{listings}
\usepackage{color}


\definecolor{dkgreen}{rgb}{0,0.6,0}
\definecolor{gray}{rgb}{0.5,0.5,0.5}
\definecolor{mauve}{rgb}{0.58,0,0.82}

\lstset{
  language=Python,                
  basicstyle=\footnotesize,           
  numbers=left,                   
  numberstyle=\tiny\color{gray},  
  stepnumber=2,                             
  numbersep=5pt,                  
  backgroundcolor=\color{white},    
  showspaces=false,               
  showstringspaces=false,         
  showtabs=false,                 
  frame=single,                   
  rulecolor=\color{black},        
  tabsize=2,                      
  captionpos=b,                   
  breaklines=true,                
  breakatwhitespace=false,        
  title=\lstname,                               
  keywordstyle=\color{blue},          
  commentstyle=\color{dkgreen},       
  stringstyle=\color{mauve},     
}

\pagestyle{empty}

\headheight 40mm      %
\oddsidemargin 2.0mm  %
\evensidemargin 2.0mm %
\topmargin -40mm      %
\textheight 250mm     %
\textwidth 160mm      %
%
\newcounter{execs}
\setcounter{execs}{0}
\newcommand{\exec}[0]{\addtocounter{execs}{1}\item[\textbf{\arabic{execs}.}]}

\fancypagestyle{first}
{
\pagestyle{fancy}
}
%%%%%%%%%%%%%%%%%%%%%%%%%%%%%%%%%%%%%%%%%%%%%%%%%%%%%%%%
%%%%%%%%%%%%%%%%%%%%%%%%%%%%%%%%%%%%%%%%%%%%%%%%%%%%%%%%
% PLEASE, EDIT THIS!
\fancyhead[LO]{\small $3^a$ Lista \\ 
                DCC008 - Cálculo Numérico  \\
                \textbf{Entrega: 14 de Outubro de 2018} }

\fancyhead[RO]{\small Universidade Federal de Juiz de Fora - UFJF \\ 
                Departamento de Ciência da Computação \\
               \textit{Nome: Thiago de Almeida}\\
               \textit{Nome: Renan Nunes}}


\begin{document}
\thispagestyle{first}

\noindent \textbf{Obs1.:} antes de resolver os exercícios abaixo, teste cada um dos métodos implementados resolvendo sistemas lineares simples de dimensão $3\times 3$ (use um exemplo dos Slides).

\noindent \textbf{Obs2.:} utilize precisão dupla.
\begin{itemize}

\exec Resolva o sistema gerado pela questão 2 da $1^a$ Lista utilizando métodos diretos (Thomas, Gauss, LU e Cholesky) e métodos iterativos (Jacobi e Gauss-Seidel). Para este estudo, considere:

\begin{itemize}
\item Sistemas lineares de diferentes dimensões (ex. 1000, 5000, 10000) a partir do problema resolvido na $1^a$ Lista variando o número de elementos; 
\item A aplicação do critério das linhas para determinar se os métodos iterativos convergem neste caso;
\item O tempo de processamento de cada método, direto ou iterativo, para a resolução do sistema e monte uma tabela;
\item Apresente o critério de parada e a tolerância utilizada para os métodos iterativos;
\item Comente os resultados obtidos.
\end{itemize}


\text O critério de Parada Utilizado nas implementações é o Erro dos métodos numéricos, com o critério aceitavel de $0,01$. 

\subsection*{Python}
\begin{lstlisting}
def normaInfinito(self, x):
        size = len(x)
        maximo = abs(x[0])   
        
        for i in range(size):
            temp = abs(x[i])        
            if(temp > maximo):
                    maximo = temp
                
        return maximo
        
    def distanciaInfinito(self, x1, x2):
        if(len(x1) != len(x2)):
            print("O tamanho dos vetores x1 e x2 precisa ser o mesmo")
            return 0
            
        size = len(x1)
        dist = abs(x1[0] - x2[0])  
        
        for i in range(size):
            temp = abs(x1[0] - x2[0])
            if(dist > temp):
                temp = dist
                
        return dist
            
    def calculaErro(self, x_prox, x_atual):
        return self.distanciaInfinito(x_prox, x_atual) / self.normaInfinito(x_prox)
    def erroResidual(self, M, X, B):
        size = len(M[0])
        erroRet = []
        for i in range(size):
            valor = 0
            for j in range(size):
                valor += M[i][j] * X[j]

            erroRet.append(abs(valor - B[i]))

        return [erroRet, max(erroRet)]
\end{lstlisting}

\newpage

\text Para uma Matriz 100x100:

\begin{table}[h]
\centering
  \begin{tabular}{l||l|lll}
    $Metodo$ & $Passos$ & $Tempo$ $de$ $Execução$ \\
    \hline
    Thomas (direto) & 9997 & 0:00:00.001998 \\
    
    Gauss (direto) & 328251 & 0:00:00.084985 \\
    
    Gauss com Pivoteamento Parcial & 338052 & 0:00:00.093985 \\
    
    Gauss-Seidel (interativo) &  &  \\
    
    LU &  \\
    
    Choleski & \\
    
    Jacobi (interativo) & & 0:00:00.020997 \\
    \hline
  \end{tabular}
  \caption{Matriz 100x100}
\end{table}


\begin{figure}[!htb]
\includegraphics [width=7cm,height=7cm]{G100part.png}
\includegraphics [width=7cm,height=7cm]{T100part}
\includegraphics [width=7cm,height=7cm]{GP100part.png}
\includegraphics [width=7cm,height=7cm]{J100part.png}
\end{figure}

\newpage

\text Para uma matriz 1000x1000:

\begin{table}[h]
\centering
  \begin{tabular}{l||l|lll}
    $Metodo$ & $Passos$ & $Tempo$ $de$ $Execução$ \\
    \hline
    Thomas (direto) & 999997  & 0:00:00.232961 \\
    
    Gauss (direto) & 332832501 & 0:01:37.112041 \\
    
    Gauss com Pivoteamento Parcial & 333830502 & 0:01:32.923215 \\
    
    Gauss-Seidel (interativo) &  &  0:00:01.087303 \\
    
    LU &  \\
    
    Choleski & \\
    
    Jacobi (interativo) & &  \\
    \hline
  \end{tabular}
  \caption{Matriz 1000x1000}
\end{table}

\begin{figure}[!htb]
\includegraphics[width=7cm,height=7cm]{G1000part.png}
\includegraphics [width=7cm,height=7cm]{T1000part}
\includegraphics [width=7cm,height=7cm]{GP1000part.png}
\includegraphics [width=7cm,height=7cm]{J1000part.png}
\end{figure}


\newpage

\text Para uma matriz 2000x2000:

\begin{table}[h]
\centering
  \begin{tabular}{l||l|lll}
    $Metodo$ & $Passos$ & $Tempo$ $de$ $Execução$ \\
    \hline
    Thomas (direto) & 2664665001  & 0:11:12.355602 \\
    
    Gauss (direto) & 332832501 & 0:01:37.112041 \\
    
    Gauss com Pivoteamento Parcial & 2668661002 & 0:12:07.371103 \\
    
    Gauss-Seidel (interativo) & 59940015  &  0:00:15.697584 \\
    
    LU &  \\
    
    Choleski & \\
    
    Jacobi (interativo) & &  \\
    \hline
  \end{tabular}
  \caption{Matriz 2000x2000}
\end{table}


\begin{figure}[!htb]
\includegraphics[width=7cm,height=7cm]{G2000part.png}
\includegraphics [width=7cm,height=7cm]{T2000part}
\includegraphics [width=7cm,height=7cm]{GP2000part.png}
\includegraphics [width=7cm,height=7cm]{J2000part.png}
\end{figure}

\newpage
\exec Seja o sistema linear:
$$
\mathbf{A}\mathbf{x}=\mathbf{b}
$$
onde,
$$
\mathbf{A} = A_{i,j} = \dfrac{1}{i+j+1} \quad \mbox{e} \quad \mathbf{b}=b_i =\dfrac{1}{i+n+1} .
$$
Supondo a matriz $\mathbf{A}_{n\times n}$, com diferentes dimensões $n$ (ex. $n = 10, 100, 1000$),  faça:

\begin{itemize}
\item Resolva utilizando o método de eliminação de Gauss \textbf{sem} e \textbf{com} pivoteamento e a decomposição LU.

\item Determine o erro cometido, por cada um dos métodos utilizados, através do resíduo calculado na norma do máximo, dado por:
$$
\|\mathbf{A}\mathbf{x} - \mathbf{b}\|_{\infty} = \max_{1\leq i\leq n}|A_{i,j}x_{i} - b_{i}|, \quad \forall j \in [1,n]
$$
onde $x_{i}$ é o vetor solução. Compare e discuta os resultados.

\end{itemize}


\end{itemize}

\text Supondo matrizes de dimensões 10,100 e 1000;
Para 100:
\begin{table}[h]
\centering
  \begin{tabular}{l||l|l|l}
     & $ Gauss$ & $Gauss$ $com$ $Pivoteamento$ $Parcial$ & $LU$ \\
    \hline
    
    PASSOS & 338250 & 348250 & \\
    
    TEMPO DE EXECUÇÃO &  0:00:00.156041 & 0:00:00.100983 & \\
    
    \hline
  \end{tabular}
  \caption{Comparação de resultados de diferentes dimensões}
\end{table}

\begin{table}[h]
\centering
  \begin{tabular}{l||l|l|l}
     & $ Gauss$ & $Gauss$ $com$ $Pivoteamento$ $Parcial$ & $LU$ \\
    \hline
    
    Erro Max & 59153715.99267997 & 3.9898639947466563e-17 & \\
    
    
    \hline
  \end{tabular}
  \caption{Comparação de erros}
\end{table}

\begin{figure}[!htb]
\includegraphics[width=7cm,height=7cm]{EGauss100part.png}
\includegraphics [width=7cm,height=7cm]{EGaussP100part.png}
\includegraphics [width=7cm,height=7cm]{}
\end{figure}

\newpage
Para 10 elementos:

\begin{table}[h]
\centering
  \begin{tabular}{l||l|l|l}
     & $ Gauss$ & $Gauss$ $com$ $Pivoteamento$ $Parcial$ & $LU$ \\
    \hline
    
    PASSOS & 375 & 475 & \\
    
    TEMPO DE EXECUÇÃO & 0:00:00.000989 & 0:00:00.000997 & \\
    
    \hline
  \end{tabular}
  \caption{Comparação de resultados de diferentes dimensões}
\end{table}

\begin{table}[h]
\centering
  \begin{tabular}{l||l|l|l}
     & $ Gauss$ & $Gauss$ $com$ $Pivoteamento$ $Parcial$ & $LU$ \\
    \hline
    
    Erro Max & 27717.293488028645 & 2.7755575615628914e-16 & \\
    
    
    \hline
  \end{tabular}
  \caption{Comparação de erros}
\end{table}

\begin{figure}[!htb]
\includegraphics[width=7cm,height=7cm]{EGauss10part.png}
\includegraphics [width=7cm,height=7cm]{EGaussP10part.png}
\includegraphics [width=7cm,height=7cm]{}
\end{figure}

\newpage
Para 1000 elementos:

\begin{table}[h]
\centering
  \begin{tabular}{l||l|l|l}
     & $ Gauss$ & $Gauss$ $com$ $Pivoteamento$ $Parcial$ & $LU$ \\
    \hline
    
    PASSOS & 333832500 & 334832500 & \\
    
    TEMPO DE EXECUÇÃO &  0:01:33.369046 & 0:01:27.377436 & \\
    
    \hline
  \end{tabular}
  \caption{Comparação de resultados de diferentes dimensões}
\end{table}

\begin{table}[h]
\centering
  \begin{tabular}{l||l|l|l}
     & $ Gauss$ & $Gauss$ $com$ $Pivoteamento$ $Parcial$ & $LU$ \\
    \hline
    
    Erro Max & 140669356.34659576 & 1.9081958235744878e-17 & \\
    
    
    \hline
  \end{tabular}
  \caption{Comparação de erros}
\end{table}

\begin{figure}[!htb]
\includegraphics[width=7cm,height=7cm]{EGauss1000part.png}
\includegraphics [width=7cm,height=7cm]{EGaussP1000part.png}
\includegraphics [width=7cm,height=7cm]{}
\end{figure}


\end{document}
